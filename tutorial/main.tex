\documentclass[12pt]{report}
\usepackage[utf8]{inputenc}

\usepackage{parskip}
\usepackage{graphicx}
\usepackage[hidelinks]{hyperref}
\usepackage{amsmath}
\usepackage{lipsum}
\usepackage{booktabs}




\title{Latex Tutorial}
\author{Max Fatouros}
\date{\today}

\begin{document}

\maketitle

\tableofcontents
\newpage


\section{Text}\label{sec:text}
    Write some text.

    % Paragraphs are seperated by a newline. To have it formatted like it is in this document you must put \usepackage{parskip} near the top of the document.
    Write this in a new paragraph.

    
    \begin{itemize}
        \item first item
        \item second item
        \item third
    \end{itemize}

    \begin{enumerate}
        \item a
        \item b
        \item c
    \end{enumerate}


\section{Math}\label{sec:math}
    Math can be put inline as part of the sentence like this $E = mc^2$. Or it can be put on its own line in "display mode":
    \[
        E = \gamma m_0c^2 = \sqrt{m_0^2c^4 + p^2c^2}
    \]
    You can also label equations
    \begin{equation}\label{eq:energy-momentum}
        E^2 = m^2c^4 + p^2c^2
    \end{equation}
    Can do multiline equations:
    \begin{align*}
        \sum^N_{i=1} i &= 1 + 2 + 3 + 4 + \cdots + N\\
                       &= \frac{N(N+1)}{2}
    \end{align*}

\section{Figures and Tables}
    \lipsum[1]
    
    I want to place an image below this text
    % h --- here
    % t --- top
    % b --- bottom
    % ! --- more aggressive placement
    \begin{figure}[h!]
        \centering
        \includegraphics[width=\textwidth]{kitten.png}
        \caption{Make sure to wrap the image and the caption in a figure environment.}
        \label{fig:kitten}
    \end{figure}

    And above this.

    Similarly, we can add tables
    \begin{table}[h]
        \centering
        % l,c,r defined text alignment
        \begin{tabular}{|c|c|}
            \hline
            \textbf{A} & \textbf{B}\\
            \hline
            1 & 2 \\
            \hline
            3 & 4 \\
            \hline
        \end{tabular}
        
        \caption{table}
        \label{tab:my_label}
    \end{table}

    \lipsum[1]
    % \usepackage{booktabs}
    % Generated with some online tool
    \begin{table}[h]
        \centering
        
        \begin{tabular}{lllll}
            \toprule
            A  & B  & C  & D  & E  \\
            \midrule
            1  & 2  & 3  & 4  & 5  \\
            6  & 7  & 8  & 9  & 10 \\
            11 & 12 & 13 & 14 & 16 \\
            \bottomrule
        \end{tabular}
        
        \caption{Better table}
        \label{tab:better_table}
        
    \end{table}



\section{References}
    Everything we've given a label to can now be referenced. For example I can reference Section \ref{sec:text}, any numbered and labeled equations (Eq. \ref{eq:energy-momentum}), and figures or tables (Fig. \ref{fig:kitten}).
    
    






\end{document}
